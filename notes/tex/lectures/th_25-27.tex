\section{Gestione delle Variabili}

La corretta gestione e codifica delle variabili è un passo fondamentale nella
costruzione di qualsiasi modello, inclusa la regressione lineare. Le variabili
possono essere di diverse tipologie, e ciascuna richiede un trattamento
specifico.

Le variabili di ingresso si possono classificare principalmente in tre
categorie:
\begin{itemize}
    \item \textbf{Numeriche}: Variabili quantitative su cui è possibile
    eseguire operazioni aritmetiche (es. età, temperatura, reddito).
    \item \textbf{Dicotomiche}: Variabili che possono assumere solo due valori
    (es. maschio/femmina, sì/no, acceso/spento).
    \item \textbf{Categoriali}: Variabili che rappresentano un gruppo o una
    categoria, a loro volta suddivisibili in nominali e ordinali.
\end{itemize}

\subsection{Variabili Dicotomiche}
Le variabili dicotomiche, che rappresentano due sole categorie, sono un caso
speciale ma molto comune di variabili categoriali.

\paragraph{Come Variabili di Ingresso}
Quando usate come predittori, vengono tipicamente codificate con i numeri 0 e 1.
In un modello di regressione:
\[ Y = \beta_0 + \beta_1 x_1 + \dots + \beta_j x_j + \dots + \epsilon \]
il coefficiente \(\beta_j\) associato a una variabile dicotomica \(x_j\) ha
un'interpretazione molto diretta: rappresenta la \textbf{differenza media nella
risposta} \(Y\) tra la categoria codificata come 1 e la categoria codificata
come 0. Ad esempio, se \(x_j\) è 1 per "femmina" e 0 per "maschio", \(\beta_j\)
misura di quanto, in media, la risposta \(Y\) per le femmine differisce da
quella per i maschi.

\begin{nota}{Variabili Dicotomiche come Risposta}{dichotomous-response-note}
Utilizzare una variabile dicotomica come variabile di risposta \(Y\) non è
appropriato per la regressione lineare standard. Per questo tipo di problema
esistono modelli specifici, tra cui:
\begin{itemize}
    \item \textbf{Analisi Discriminante}
    \item \textbf{Regressione Logistica}
\end{itemize}
\end{nota}

\subsection{Variabili Categoriali}
Le variabili categoriali richiedono una codifica specifica per poter essere
incluse in un modello di regressione.

\paragraph{Variabili Ordinali vs. Nominali}
È importante distinguere tra:
\begin{itemize}
    \item \textbf{Variabili Nominali}: Le categorie non hanno un ordinamento
    intrinseco. Esempi sono le stagioni (primavera, estate, autunno, inverno),
    le regioni d'Italia o il tipo di carburante.
    \item \textbf{Variabili Ordinali}: Le categorie possiedono un ordine
    naturale. Esempi sono il titolo di studio (medie, maturità, laurea), una
    classifica (I, II, III) o la gravità di un sintomo.
\end{itemize}
Le variabili ordinali possono essere codificate numericamente (es. 1, 2, 3...)
rispettando l'ordine, ma bisogna essere consapevoli che questa scelta impone una
distanza uniforme tra i livelli, il che potrebbe non essere sempre corretto.

\paragraph{Codifica One-Hot per Variabili Nominali}
Per inserire una variabile categoriale nominale in un modello di regressione, la
tecnica standard è la \textbf{codifica one-hot}, nota anche come creazione di
\textit{variabili dummy}.

\begin{definizione}{Codifica One-Hot con Categoria di Riferimento}{one-hot-def}
Una variabile categoriale con \(k\) livelli viene trasformata in \(k-1\) nuove
variabili dicotomiche (0/1). Un livello viene escluso e scelto come
\textbf{categoria di riferimento} (o \textit{default}) per evitare una perfetta
multicollinearità.
\end{definizione}

\begin{esempio}{Codifica delle stagioni}{seasons-example}
Consideriamo la variabile "stagione" con 4 livelli: primavera, estate, autunno,
inverno. Scegliamo "inverno" come categoria di riferimento. Creiamo 3 nuove
variabili dummy:
\begin{itemize}
    \item \(x_{\text{primavera}}\): vale 1 se la stagione è primavera, 0
    altrimenti.
    \item \(x_{\text{estate}}\): vale 1 se la stagione è estate, 0 altrimenti.
    \item \(x_{\text{autunno}}\): vale 1 se la stagione è autunno, 0
    altrimenti.
\end{itemize}
Il modello di regressione diventerà: \( Y = \beta_0 + \beta_1
x_{\text{primavera}} + \beta_2 x_{\text{estate}} + \beta_3 x_{\text{autunno}} +
\dots + \epsilon \).
\end{esempio}

L'interpretazione dei coefficienti è cruciale: ogni coefficiente \(\beta_j\)
misura l'effetto medio sulla risposta \(Y\) di quella categoria \textbf{rispetto
alla categoria di riferimento}.
\begin{itemize}
    \item Per l'inverno (riferimento), tutte le dummy sono 0 e il valore atteso
    di Y è legato a \(\beta_0\).
    \item Per la primavera, il valore atteso di Y è legato a \(\beta_0 +
    \beta_1\). Quindi, \(\beta_1\) rappresenta la differenza media in \(Y\) tra
    la primavera e l'inverno.
\end{itemize}

\begin{nota}{Variabili Categoriali come Risposta}{categorical-response-note}
Utilizzare una variabile dicotomica o categoriale come variabile di risposta
\(Y\) in una regressione lineare standard è problematico. Per questi casi,
esistono modelli più appropriati come la \textbf{regressione logistica}, gli
alberi di classificazione, le random forest o le reti neurali.
\end{nota}

\subsection{Interpretazione e Test sui Coefficienti delle Variabili Dummy}

Una volta inserite le variabili dummy nel modello, è fondamentale saper
interpretare correttamente i loro coefficienti e capire come testare ipotesi
specifiche.

\begin{esempio}{Interpretazione dei coefficienti delle
stagioni}{seasons-coeffs-ex}
Riprendendo il modello delle stagioni con "inverno" come riferimento:
\[ Y = \beta_0 + \beta_1 x_{\text{primavera}} + \beta_2 x_{\text{estate}} +
\beta_3 x_{\text{autunno}} + \epsilon \]
Supponiamo che la stima del modello fornisca i seguenti coefficienti:
\(\beta_0=16.4\), \(\beta_1=5.4\), \(\beta_2=6.7\), \(\beta_3=1.8\).
Il valore medio della risposta \(Y\) per ciascuna stagione è:
\begin{itemize}
    \item \textbf{Inverno (riferimento)}: \( E[Y | \text{inverno}] = \beta_0 =
    16.4 \)
    \item \textbf{Primavera}: \( E[Y | \text{primavera}] = \beta_0 + \beta_1 =
    16.4 + 5.4 = 21.8 \)
    \item \textbf{Estate}: \( E[Y | \text{estate}] = \beta_0 + \beta_2 = 16.4 +
    6.7 = 23.1 \)
    \item \textbf{Autunno}: \( E[Y | \text{autunno}] = \beta_0 + \beta_3 = 16.4
    + 1.8 = 18.2 \)
\end{itemize}
Il test t standard su un coefficiente, ad esempio \(H_0: \beta_3=0\), verifica
se c'è una differenza significativa tra l'autunno e l'inverno. Se non si
rifiuta l'ipotesi nulla, si conclude che le due categorie sono statisticamente
indistinguibili.
\end{esempio}

\paragraph{Confronto tra Categorie non di Riferimento}
Per confrontare due categorie non di riferimento (es. primavera vs. estate), non
è sufficiente guardare i singoli p-value. Bisogna testare l'ipotesi \(H_0:
\beta_1 = \beta_2\).

\begin{esercizio}{HW: Testare la differenza tra primavera ed
estate}{ex:compare-dummies}
Come si può testare l'ipotesi \(H_0: \beta_1 = \beta_2\)? Ci sono due approcci.
\begin{dimostrazione}{}{}
\begin{enumerate}
    \item \textbf{Test Manuale (Test Lineare Generale)}: L'ipotesi è
    equivalente a \(H_0: \beta_1 - \beta_2 = 0\). Si può costruire una
    statistica t per questa combinazione lineare di coefficienti:
    \[ T = \frac{\hat{\beta}_1 - \hat{\beta}_2}{\text{SE}(\hat{\beta}_1 -
    \hat{\beta}_2)} \]
    dove l'errore standard al denominatore si calcola dalla matrice di
    varianza-covarianza dei coefficienti:
    \[ \text{SE}(\hat{\beta}_1 - \hat{\beta}_2) =
    \sqrt{\text{Var}(\hat{\beta}_1) + \text{Var}(\hat{\beta}_2) -
    2\text{Cov}(\hat{\beta}_1, \hat{\beta}_2)} \]
    Questo test permette di ottenere un p-value per l'ipotesi di uguaglianza.

    \item \textbf{Cambiare la Categoria di Riferimento}: Un metodo più semplice
    e pratico consiste nel modificare il modello, scegliendo una delle categorie
    da confrontare (es. "estate") come nuovo livello di riferimento. A questo
    punto si riesegue la regressione. Il nuovo coefficiente associato alla
    variabile "primavera" misurerà direttamente la differenza media di \(Y\)
    rispetto all'"estate", e il suo test t standard fornirà il p-value
    desiderato.
\end{enumerate}
\end{dimostrazione}
\end{esercizio}

\subsection{Criticità e Note Pratiche}

\paragraph{Interazione con le Procedure Stepwise}
\begin{itemize}
    \item \textbf{Stepwise Backward}: Se la procedura elimina una variabile
    dummy (es. \(x_{\text{autunno}}\)), significa che il test non ha trovato una
    differenza significativa tra quella categoria e la categoria di riferimento.
    L'effetto è che le due categorie vengono fuse o "collassate".
    \item \textbf{Stepwise Forward}: La procedura sceglie autonomamente quale
    categoria usare implicitamente come riferimento, e non è detto che la
    scelta sia ottimale. Potrebbe non accorgersi che due categorie (nessuna
    delle quali è il riferimento) sono molto simili e andrebbero accorpate.
\end{itemize}

\begin{nota}{Attenzione}{dummy-vars-warning}
\begin{itemize}
    \item \textbf{Aumento di \(p\)}: L'uso della codifica one-hot può aumentare
    drasticamente il numero di predittori \(p\), specialmente per variabili con
    molte categorie. Questo peggiora il rapporto \(n/p\) e può rendere il
    modello instabile. Per risolvere il problema, si possono accorpare
    manualmente alcune categorie in macro-categorie più generali e
    significative.
    \item \textbf{Non-Linearità}: Per le variabili dummy (0/1), l'aggiunta di
    potenze (es. \(x^2\)) è inutile, poiché \(x^2=x\). Similmente,
    l'interazione tra dummy della stessa variabile categoriale (es.
    \(x_{\text{primavera}} \cdot x_{\text{estate}}\)) è sempre zero ed è
    quindi inutile. Sono invece utili le interazioni tra dummy di variabili
    categoriali diverse.
\end{itemize}
\end{nota}

\subsection{Gestione delle Variabili Numeriche}

Anche le variabili numeriche possono essere suddivise in base alla loro natura,
e questo influenza la decisione di trasformarle o meno prima di inserirle in un
modello.

\paragraph{Variabili di tipo "Differenza" vs. "Rapporto"}
\begin{itemize}
    \item \textbf{Tipo "Differenza"}: Sono variabili per cui le differenze
    additive sono significative e interpretabili (es. Quoziente Intellettivo,
    statura, temperatura). Generalmente, queste variabili vengono inserite nel
    modello così come sono, in un'ottica additiva del tipo \( Y = \beta_0 +
    \beta_1 x_1 + \dots + \epsilon \).
    \item \textbf{Tipo "Rapporto"}: Sono variabili tipicamente positive che
    possono coprire diversi ordini di grandezza, per le quali i rapporti (e
    quindi le variazioni percentuali) sono più significativi delle differenze
    assolute (es. reddito, popolazione di una città, intensità sonora). Per
    queste variabili è spesso opportuno applicare una trasformazione
    logaritmica.
\end{itemize}

\begin{nota}{Trasformazioni Lineari}{linear-transform-note}
Una trasformazione lineare della variabile di risposta (\(\tilde{Y} = mY+q\))
non cambia la sostanza della regressione. I coefficienti vengono riscalati
(\(\tilde{\beta}_j = m\beta_j\)), ma i risultati dei test di ipotesi (p-value,
\(\alpha^*\)) e gli indici di bontà del modello (\(R_D^2, R_A^2\)) rimangono
invariati.
\end{nota}
Le trasformazioni non-lineari, invece, possono essere usate per migliorare le
proprietà distributive di una variabile. Ad esempio, se un predittore \(x\) ha
una distribuzione asimmetrica a destra (\textit{right-skewed}), applicare una
trasformazione come la radice quadrata (\(\sqrt{x}\)) o il logaritmo
(\(\ln(x)\)) può renderne la distribuzione più simmetrica, aiutando a
soddisfare le assunzioni del modello di regressione lineare.

\subsection{Modelli Logaritmici e Interpretazione}

Una delle trasformazioni più potenti e comuni è l'uso del logaritmo sulla
variabile di risposta, specialmente quando questa è di tipo "rapporto".

\paragraph{Il Modello Log-Level}
Quando si modella il logaritmo di Y (modello log-level), la relazione diventa:
\[ \ln(Y) = \beta_0 + \beta_1 x_1 + \dots + \epsilon \]
Questo modello è ancora lineare nei parametri, ma implica una relazione
\textbf{moltiplicativa} sulla scala originale di Y:
\[ Y = e^{\beta_0} \cdot e^{\beta_1 x_1} \cdot \dots \cdot e^{\epsilon} \]
Questo approccio è preferibile quando si ritiene che i predittori abbiano un
effetto percentuale, e non assoluto, sulla risposta.

\begin{esempio}{Interpretazione del coefficiente in un modello
Log-Level}{log-level-interp-ex}
Supponiamo di modellare lo stipendio (\(Y\)) in funzione di una variabile
dicotomica \(x_1\) (1 se una persona ha frequentato un certo corso, 0
altrimenti). Il modello è \(\ln(Y) = \beta_0 + \beta_1 x_1\).
\begin{itemize}
    \item Per chi non ha frequentato il corso (\(x_1=0\)): \( \ln(Y_0) = \beta_0
    \implies Y_0 = e^{\beta_0} \)
    \item Per chi ha frequentato il corso (\(x_1=1\)): \( \ln(Y_1) = \beta_0 +
    \beta_1 \implies Y_1 = e^{\beta_0}e^{\beta_1} = Y_0 \cdot e^{\beta_1} \)
\end{itemize}
Il coefficiente \(\beta_1\) non rappresenta più una differenza additiva. Il
fattore \(e^{\beta_1}\) è un \textbf{fattore moltiplicativo}. Se la stima del
modello fornisce \(\hat{\beta}_1\) tale che \(e^{\hat{\beta}_1} = 1.40\),
significa che frequentare il corso è associato a un aumento dello stipendio del
40\% (\(Y_1 = 1.40 \cdot Y_0\)). La variazione percentuale si calcola come
\((e^{\hat{\beta}_1} - 1) \times 100\%\).
\end{esempio}


\section{Regressione Logistica}

La regressione logistica è un modello statistico utilizzato quando la variabile
di risposta \(Y\) è categoriale. A differenza della regressione lineare, non
modella direttamente il valore della risposta, ma la \textbf{probabilità} che
la risposta appartenga a una determinata categoria.

\subsection{Regressione Logistica Binomiale}
Il caso più comune è quello in cui la variabile di risposta è
\textbf{dicotomica}, ovvero può assumere solo due valori (es. 0/1,
successo/fallimento, malato/sano).

\paragraph{Il Modello}
L'obiettivo è modellare la probabilità che la risposta sia 1, dato un set di
predittori \(X\). Poiché una probabilità deve essere compresa tra 0 e 1, si
utilizza la \textbf{funzione logistica (o sigmoide)} per mappare la combinazione
lineare dei predittori (chiamata \textbf{logit}) nell'intervallo (0, 1).

\begin{definizione}{Funzione Logistica (Sigmoide)}{sigmoid-def}
Il modello lega i predittori alla probabilità di successo \(p(X) = P(Y=1 | X)\)
attraverso la seguente relazione:
\[
    p(X) = \sigma(z) = \frac{1}{1+e^{-z}}
\]
dove \(z = \beta_0 + \beta_1 x_1 + \dots + \beta_p x_p\) è il logit.
\end{definizione}

\paragraph{Stima tramite Massima Verosimiglianza (MLE)}
Per stimare i coefficienti \(\beta_j\), si assume che ogni osservazione \(y_i\)
sia il risultato di una \textbf{prova Bernoulliana}. La probabilità di successo
di questa prova, \(p_i\), è data dal modello logistico. La funzione di massa di
probabilità per una singola osservazione è quindi:
\[ P(Y=y_i | X_i) = p_i^{y_i}(1-p_i)^{1-y_i} \]
La verosimiglianza (\textit{likelihood}) per l'intero dataset di \(N\)
osservazioni indipendenti è il prodotto delle singole probabilità. Per
semplicità, si massimizza la sua versione logaritmica (la
\textbf{log-verosimiglianza}):
\begin{align*}
    l(\beta) &= \log \left( \prod_{i=1}^{N} p_i^{y_i}(1-p_i)^{1-y_i} \right) \\
    &= \sum_{i=1}^{N} \left[ y_i \log(p_i) + (1-y_i) \log(1-p_i) \right]
\end{align*}
Questa espressione è esattamente l'opposto della funzione di costo di
\textbf{cross-entropy}. Pertanto, massimizzare la verosimiglianza è equivalente
a minimizzare la cross-entropy.

\begin{nota}{Inferenza sui Coefficienti}{logistic-inference-note}
A differenza della regressione lineare, i coefficienti \(\beta_j\) non seguono
una distribuzione t di Student. Tuttavia, si basano su approssimazioni alla
distribuzione normale per grandi campioni. L'inferenza statistica e la selezione
delle variabili sono quindi possibili e si basano comunemente su:
\begin{itemize}
    \item \textbf{Wald Test}: Un test simile al t-test per verificare l'ipotesi
    nulla che un singolo coefficiente sia uguale a zero.
    \item \textbf{Likelihood-Ratio Test (LRT)}: Un test più robusto per
    confrontare modelli annidati e valutare la significatività di un gruppo di
    variabili.
\end{itemize}
\end{nota}

Una volta addestrato, il modello si usa per prevedere la probabilità di
successo per nuove osservazioni.

\subsection{Regressione Logistica Multinomiale}
Questa è la generalizzazione del modello a casi in cui la variabile di risposta
ha più di due categorie (es. \(k \in \{1, \dots, m\}\)).

\paragraph{Il Modello}
Il modello predice un vettore di probabilità \(\pi = (\pi_1, \pi_2, \dots,
\pi_m)\), dove \(\pi_k\) è la probabilità che l'osservazione appartenga alla
categoria \(k\), con il vincolo che \(\sum \pi_k = 1\). Per fare ciò, si usa la
\textbf{funzione softmax}, che è la generalizzazione della sigmoide.

\begin{definizione}{Funzione Softmax}{softmax-def}
Per ogni categoria \(k\) si calcola un logit \(z_k\) con un set di parametri
dedicato \(w_k\):
\[
    z_k = w_{k0} + w_{k1}x_1 + \dots + w_{kp}x_p
\]
La probabilità per la categoria \(k\) è data dalla funzione softmax, che
normalizza i logit:
\[
    \pi_k = \text{softmax}(z_k) = \frac{e^{z_k}}{\sum_{j=1}^{m} e^{z_j}}
\]
\end{definizione}

\paragraph{Stima del Modello}
La stima dei parametri \(w\) avviene, anche in questo caso, massimizzando la
verosimiglianza, la cui derivazione è discussa in \ref{likelihood:mult_cond}.
Ciò corrisponde a minimizzare la \textbf{loss di cross-entropy categoriale},
che per una singola osservazione è:
\[ H(y_i, \pi_i) = - \sum_{k=1}^{m} y_{ik} \log(\pi_{ik}) \]
dove \(y_i\) è il vettore one-hot della classe vera (es. `[0, 1, 0]`) e
\(\pi_i\) è il vettore delle probabilità predette. La loss totale da
minimizzare è la media di questa quantità su tutto il dataset.

\begin{nota}{Nota Pratica sull'Implementazione della
Loss}{loss-implementation-note}
Sebbene la cross-entropy sia concettualmente definita tra la distribuzione di
probabilità vera (\(y_i\)) e quella predetta (\(\pi_i\)), nella pratica è
sconsigliato calcolare manualmente la funzione softmax per ottenere le
probabilità \(\pi_i\) e poi passarle alla funzione di loss.

Il calcolo esplicito dell'esponenziale nella funzione softmax (\(e^{z_k}\)) può
portare a instabilità numerica (errori di \textit{overflow} o
\textit{underflow}) quando i valori dei logit \(z_k\) sono molto grandi o molto
piccoli.

Per questo motivo, le librerie software di machine learning (come TensorFlow,
PyTorch, Scikit-learn) offrono implementazioni della cross-entropy loss che
prendono in input direttamente i \textbf{logit} grezzi. Queste funzioni
utilizzano internamente degli accorgimenti matematici (come il trucco
"Log-Sum-Exp") per calcolare la loss in modo numericamente stabile.

Pertanto, la regola pratica è: \textbf{passare sempre i logit, non le
probabilità, alla funzione di loss fornita dalla libreria}.
\end{nota}



\section{Analisi della Varianza (ANOVA)}

L'Analisi della Varianza (ANOVA) è una tecnica statistica che può essere vista
come una variante della regressione lineare, utilizzata quando le variabili di
ingresso sono \textbf{categoriali}. Il modello analizza la relazione tra una o
più variabili categoriali indipendenti e una variabile dipendente numerica,
assumendo che l'errore sia additivo, Gaussiano e omoschedastico.

I principali tipi di ANOVA sono:
\begin{itemize}
    \item \textbf{ANOVA a una via}: utilizzata quando si ha una sola variabile
    di ingresso categoriale.
    \item \textbf{ANOVA a due vie}: utilizzata con due variabili di ingresso
    categoriali. Si distingue tra disegni con o senza repliche (più di
    un'osservazione per ogni combinazione di categorie).
\end{itemize}

\subsection{ANOVA a Una Via (One-Way ANOVA)}
Questo è il modello ANOVA più semplice. Si usa per confrontare le medie di tre
o più gruppi (categorie) definiti da un singolo fattore.

\paragraph{Il Modello Statistico}
L'idea centrale è che ogni categoria della variabile di ingresso possa avere
una propria media per la variabile di risposta. Si assume che la varianza
all'interno di ogni gruppo sia la stessa per tutti (omoschedasticità).

\begin{definizione}{Modello ANOVA a una via}{oneway-anova-def}
Sia \(Y_{ij}\) la \(j\)-esima osservazione nel \(i\)-esimo gruppo, dove \(i=1,
\dots, m\) (numero di gruppi) e \(j=1, \dots, n_i\) (numerosità del gruppo
\(i\)). Il modello statistico è:
\[ Y_{ij} \sim N(\mu_i, \sigma^2) \]
I dati sono assunti indipendenti. I parametri incogniti del modello sono le
\(m\) medie dei gruppi \(\mu_1, \mu_2, \dots, \mu_m\) e la varianza comune
\(\sigma^2\).
\end{definizione}

\paragraph{Stima dei Parametri del Modello}
La stima dei parametri incogniti del modello avviene a partire dai dati
campionari.

\begin{itemize}
    \item \textbf{Stima delle Medie (\(\mu_i\))}: La stima naturale per la media
    di ciascun gruppo è la sua media campionaria:
    \[ \hat{\mu}_i = \bar{Y}_{i*} = \frac{1}{n_i} \sum_{j=1}^{n_i} Y_{ij} \]

    \item \textbf{Stima della Varianza Comune (\(\sigma^2\))}: La stima della
    varianza comune \(\sigma^2\) è un processo a due passi.
    \begin{enumerate}
        \item \textbf{Stima per singolo gruppo}: Per prima cosa, si calcola la
        varianza campionaria per ciascun gruppo \(i\), definita come \(S_i^2\):
        \[ S_i^2 = \frac{1}{n_i-1} \sum_{j=1}^{n_i} (Y_{ij} - \bar{Y}_{i*})^2 \]
        Ognuno di questi \(S_i^2\) è uno stimatore corretto di \(\sigma^2\)
        (cioè \(E[S_i^2] = \sigma^2\)), ma utilizza solo una parte dei dati. Si
        ottengono così \(m\) stimatori diversi per lo stesso parametro.
        
        \item \textbf{Combinazione degli stimatori (Stimatore Pooled)}: Per
        ottenere una stima singola e più robusta, si combinano gli \(m\)
        stimatori in una media pesata. Lo stimatore risultante è detto
        \textbf{stimatore pooled della varianza} o \textbf{Within} (\(S_W^2\)),
        noto anche come \textbf{Mean Square Within} (MSW):
        \[ S_W^2 = \frac{\sum_{i=1}^{m} (n_i-1)S_i^2}{N-m} \]
    \end{enumerate}
\end{itemize}

\begin{definizione}{Somma dei Quadrati Entro i Gruppi (SSW)}{ssw-def}
Il numeratore dello stimatore \(S_W^2\) è la Somma dei Quadrati Entro i Gruppi
(Sum of Squares Within), o \textbf{devianza within}:
\[ SS_W = \sum_{i=1}^{m} (n_i-1)S_i^2 = \sum_{i=1}^{m} \sum_{j=1}^{n_i} (Y_{ij}
- \bar{Y}_{i*})^2 \]
Questa quantità rappresenta la variabilità totale all'interno dei gruppi e
corrisponde alla Somma dei Quadrati dei Residui (SSR) del modello ANOVA. Lo
stimatore della varianza può essere scritto come:
\[ S_W^2 = \frac{SS_W}{N-m} \]
\end{definizione}

\paragraph{Inferenza sui Singoli Parametri}
Disponendo degli stimatori e delle loro distribuzioni, è possibile costruire
test e intervalli di confidenza sui singoli parametri, come ad esempio per una
singola media \(\mu_i\) tramite la statistica \(T = \frac{\bar{Y}_{i*} -
\mu_i}{S_W / \sqrt{n_i}} \sim t(N-m)\).

\subsection{Il Test F nell'ANOVA a Una Via}
Lo scopo principale dell'ANOVA a una via è testare se le medie dei vari gruppi sono tutte uguali o se almeno una è diversa. Questo permette di determinare se la variabile categoriale ha un effetto statisticamente significativo sulla variabile di risposta.

\paragraph{Ipotesi del Test}
Il test fondamentale dell'ANOVA confronta due ipotesi:
\begin{itemize}
    \item \textbf{Ipotesi Nulla (\(H_0\))}: Tutte le medie dei gruppi sono uguali.
    \[ H_0: \mu_1 = \mu_2 = \dots = \mu_m \]
    Sotto \(H_0\), la variabile categoriale non ha alcun effetto sulla risposta \(Y\).
    \item \textbf{Ipotesi Alternativa (\(H_1\))}: Non tutte le medie sono uguali (almeno una è diversa). Sotto \(H_1\), la variabile categoriale ha un effetto su \(Y\).
\end{itemize}

\paragraph{Decomposizione della Varianza}
La logica del test F si basa sulla scomposizione della variabilità totale dei dati in due parti: la variabilità \textit{tra} i gruppi e la variabilità \textit{all'interno} dei gruppi.

\begin{definizione}{Somma dei Quadrati Totale (SST)}{sst-def}
La Somma dei Quadrati Totale misura la variabilità totale di tutte le osservazioni attorno alla media generale (\(\bar{Y}_{**}\)).
\[SS_T = \sum_{i=1}^{m}\sum_{j=1}^{n_i} (Y_{ij} - \bar{Y}_{**})^2\]
\end{definizione}

\begin{definizione}{Somma dei Quadrati Tra i Gruppi (SSB)}{ssb-def}
La Somma dei Quadrati Tra i Gruppi, o \textbf{devianza between}, misura la variabilità delle medie di ciascun gruppo attorno alla media generale. Ha \(m-1\) gradi di libertà.
\[ SS_B = \sum_{i=1}^{m} n_i (\bar{Y}_{i*} - \bar{Y}_{**})^2 \]
\end{definizione}
Queste quantità sono legate alla devianza within (\(SS_W\)) dalla relazione
fondamentale:
\[
    SS_T = SS_B + SS_W
\]

\paragraph{La Statistica del Test F}
Il test si basa sul confronto tra la varianza stimata "tra" i gruppi e quella "entro" i gruppi. Queste stime sono chiamate \textbf{Medie dei Quadrati} (Mean Squares).

\begin{definizione}{Medie dei Quadrati (MS)}{ms-def}
Le Medie dei Quadrati si ottengono dividendo le somme dei quadrati per i rispettivi gradi di libertà.
\begin{itemize}
    \item \textbf{Media dei Quadrati Tra i Gruppi}: \( MS_B = S_B^2 = \frac{SS_B}{m-1} \)
    \item \textbf{Media dei Quadrati Entro i Gruppi}: \( MS_W = S_W^2 = \frac{SS_W}{N-m} \)
\end{itemize}
L'idea è che \(MS_W\) è sempre una stima corretta della varianza \(\sigma^2\), mentre \(MS_B\) lo è solo se \(H_0\) è vera. Se \(H_0\) è falsa, \(MS_B\) tenderà ad essere più grande.
\end{definizione}

\begin{teorema}{Test F per l'ANOVA a una via}{anova-f-test-thm}
La statistica del test è il rapporto tra le due medie dei quadrati:
\[ F_{ANOVA} = \frac{MS_B}{MS_W} = \frac{S_B^2}{S_W^2} \]
Sotto l'ipotesi nulla \(H_0\), questa statistica segue una distribuzione F di Fisher con \(m-1\) e \(N-m\) gradi di libertà:
\[ F_{ANOVA} \sim F(m-1, N-m) \]
Poiché sotto \(H_1\) la statistica F tende ad assumere valori grandi, il test è \textbf{unilaterale destro}.
\end{teorema}

\subsection{Verifica delle Assunzioni e Note Pratiche}
\begin{nota}{Attenzione alle Cose Pratiche}{anova-practical-notes}
\begin{itemize}
    \item \textbf{Analisi dei Residui}: È fondamentale verificare le assunzioni
    del modello analizzando i residui (\(R_{ij} = Y_{ij} - \bar{Y}_{i*}\)).
    Graficamente, si deve controllare che siano approssimativamente Gaussiani,
    omoschedastici (varianza costante tra i gruppi), indipendenti dal predittore
    e che non presentino outlier evidenti.
    \item \textbf{Trasformazioni}: Se le assunzioni non sono soddisfatte, a
    volte una trasformazione non lineare della variabile di risposta \(Y\) (es.
    logaritmo) può risolvere il problema.
    \item \textbf{Accettazione di \(H_0\)}: Se il test F non porta a rifiutare
    l'ipotesi nulla, significa che non c'è evidenza statistica che la variabile
    categoriale influenzi la risposta. In questo caso, si potrebbe considerare
    di ignorare la suddivisione in gruppi e analizzare i dati come un unico
    campione Gaussiano i.i.d. \(Y_i \sim N(\mu, \sigma^2)\).
\end{itemize}
\end{nota}
