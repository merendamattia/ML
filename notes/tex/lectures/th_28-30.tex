\subsection{ANOVA a Due Vie (Two-Way ANOVA)}
L'ANOVA a due vie si utilizza quando si vuole analizzare l'effetto di
\textbf{due variabili categoriali} (o fattori) su una variabile di risposta
numerica. Questo modello permette di studiare non solo l'effetto individuale di
ciascun fattore (effetto principale), ma anche se l'effetto di un fattore
dipende dal livello dell'altro (effetto di interazione).

\paragraph{Notazione e Modelli}
Sia \(Y_{ijk}\) la \(k\)-esima osservazione (o replica) per il livello \(i\) del
primo fattore (righe) e il livello \(j\) del secondo fattore (colonne). La media
della cella \((i,j)\) viene scomposta per separare i diversi effetti.

\begin{definizione}{Modelli Additivo e con Interazione}{twoway-models-def}
\begin{itemize}
    \item \textbf{Modello con Interazione} (usato con repliche, \(l \ge 2\)):
    scompone la media in quattro parti:
    \[ \mu_{ij} = \mu + \alpha_i + \beta_j + \gamma_{ij} \]
    dove \(\mu\) è la media globale, \(\alpha_i\) è l'effetto principale del
    fattore di riga, \(\beta_j\) è l'effetto principale del fattore di colonna
    e \(\gamma_{ij}\) è il \textbf{termine di interazione}.
    
    \item \textbf{Modello Additivo} (usato senza repliche, \(l=1\)): assume che
    non ci sia interazione:
    \[ \mu_{ij} = \mu + \alpha_i + \beta_j \]
    In questo modello, l'effetto di un fattore è costante a tutti i livelli
    dell'altro fattore. Per garantire l'unicità del modello, si impongono
    vincoli di somma zero sugli effetti (\(\sum \alpha_i = 0, \sum \beta_j =
    0\)).
\end{itemize}
\end{definizione}

\paragraph{Analisi per il Modello Additivo (Senza Repliche)}
Quando si ha una sola osservazione per cella (\(l=1\)), si assume che il modello
sia additivo. La stima dei parametri e degli effetti si basa sulle medie di riga
(\(\bar{Y}_{i*}\)), di colonna (\(\bar{Y}_{*j}\)) e sulla media generale
(\(\bar{Y}_{**}\)).

\begin{definizione}{Stimatori degli Effetti
Principali}{main-effects-estimators-def}
Gli effetti principali sono stimati come lo scostamento delle medie parziali
dalla media generale:
\[ \hat{\alpha}_i = \bar{Y}_{i*} - \bar{Y}_{**} \qquad \text{e} \qquad
\hat{\beta}_j = \bar{Y}_{*j} - \bar{Y}_{**} \]
\end{definizione}

\begin{definizione}{Valori Previsti e Residui}{residuals-2way-def}
Il valore previsto (o stimato) dal modello per la cella \((i,j)\) è
\(\hat{Y}_{ij} = \bar{Y}_{i*} + \bar{Y}_{*j} - \bar{Y}_{**}\). I residui sono la
differenza tra i valori osservati e quelli previsti:
\[ R_{ij} = Y_{ij} - \hat{Y}_{ij} = Y_{ij} - \bar{Y}_{i*} - \bar{Y}_{*j} +
\bar{Y}_{**} \]
\end{definizione}

\begin{definizione}{Stima della Varianza d'Errore}{error-variance-2way-def}
La Somma dei Quadrati dell'Errore (\(SS_e\)) è la somma dei residui al
quadrato. Lo stimatore corretto della varianza \(\sigma^2\), indipendente dalle
ipotesi, è la Media dei Quadrati dell'Errore (\(MS_e\)):
\[ \hat{\sigma}^2 = S_e^2 = MS_e = \frac{SS_e}{(m-1)(n-1)} \quad \text{dove}
\quad SS_e = \sum_{i,j} R_{ij}^2 \]
Questa stima, scalata, segue una distribuzione Chi-Quadrato:
\(\frac{SS_e}{\sigma^2} \sim \chi^2_{(m-1)(n-1)}\).
\end{definizione}

L'analisi completa dei dati e i test di ipotesi vengono quindi riassunti nelle
seguenti tabelle.

\begin{table}[ht]
    \centering
    \caption{Tabella ANOVA a due fattori (caso senza repliche) - Scomposizione
    della Varianza}
    \begin{tabular}{l l l}
        \toprule
        \textbf{Fonte di variabilità} & \textbf{Somma di quadrati (SS)} &
        \textbf{Gradi di libertà (df)} \\
        \midrule
        Riga (Fattore 1) & \(SS_r = n \sum (\bar{Y}_{i*} - \bar{Y}_{**})^2\) &
        \(m-1\) \\
        Colonna (Fattore 2) & \(SS_c = m \sum (\bar{Y}_{*j} - \bar{Y}_{**})^2\)
        & \(n-1\) \\
        Errore & \(SS_e = \sum (Y_{ij} - \bar{Y}_{i*} - \bar{Y}_{*j} +
        \bar{Y}_{**})^2\) & \((m-1)(n-1)\) \\
        \bottomrule
    \end{tabular}
\end{table}

\begin{table}[ht]
    \centering
    \caption{Tabella ANOVA a due fattori (caso senza repliche) - Test di
    Ipotesi}
    \resizebox{\textwidth}{!}{
    \begin{tabular}{l l l l}
        \toprule
        \textbf{Ipotesi Nulla} & \textbf{Statistica del Test} & \textbf{Un test
        con significatività \(\alpha\) deve...} & \textbf{p-dei-dati se
        \(D_{ts}=v\)} \\
        \midrule
        \(H_0: \text{Tutte le } \alpha_i = 0\) & \(D_{ts} =
        \frac{SS_r/(m-1)}{SS_e/((m-1)(n-1))}\) & rifiutare \(H_0\) se \(D_{ts} >
        F_{\alpha, m-1, (m-1)(n-1)}\) & \(P(F_{m-1, (m-1)(n-1)} \ge v)\) \\
        \(H_0: \text{Tutte le } \beta_j = 0\) & \(D_{ts} =
        \frac{SS_c/(n-1)}{SS_e/((m-1)(n-1))}\) & rifiutare \(H_0\) se \(D_{ts} >
        F_{\alpha, n-1, (m-1)(n-1)}\) & \(P(F_{n-1, (m-1)(n-1)} \ge v)\) \\
        \bottomrule
    \end{tabular}}
\end{table}
\begin{teorema}{Test F per il modello additivo}{anova-f-test-2way-add-thm}
Per testare gli effetti principali si usano due distinte statistiche F, che
confrontano la varianza spiegata da ciascun fattore con la varianza residua
\(MS_e = S_e^2\).
\begin{itemize}
    \item \textbf{Test per le Righe}: \( F_r = \frac{MS_r}{S_e^2} \sim F(m-1,
    (m-1)(n-1))\)
    \item \textbf{Test per le Colonne}: \( F_c = \frac{MS_c}{S_e^2} \sim F(n-1,
    (m-1)(n-1))\)
\end{itemize}
\end{teorema}

\paragraph{Analisi per il Modello con Repliche}
La presenza di repliche (\(l \ge 2\)) è fondamentale perché permette di
stimare separatamente la media di ogni cella \(\mu_{ij}\) e, di conseguenza, di
isolare l'effetto di interazione dalla variabilità casuale.

\begin{itemize}
    \item \textbf{Stimatori delle Medie e degli Effetti}: Lo stimatore naturale
    per la media della cella \((i,j)\) è la media campionaria delle
    osservazioni in quella cella:
    \[ \hat{\mu}_{ij} = \bar{Y}_{ij*} = \frac{1}{l}\sum_{k=1}^{l}Y_{ijk} \]
    Da questi si ricavano gli stimatori per gli effetti: \(\hat{\alpha}_i =
    \bar{Y}_{i**}-\bar{Y}_{***}\), \(\hat{\beta}_j =
    \bar{Y}_{*j*}-\bar{Y}_{***}\) e \(\hat{\gamma}_{ij} = \bar{Y}_{ij*} -
    \bar{Y}_{i**} - \bar{Y}_{*j*} + \bar{Y}_{***} \).
    \item \textbf{Residui e Stima della Varianza d'Errore}: I residui sono la
    variabilità interna a ciascuna cella, attorno alla propria media:
    \[ R_{ijk} = Y_{ijk} - \bar{Y}_{ij*} \]
    La Somma dei Quadrati dell'Errore (\(SS_e\)) è la somma dei residui al
    quadrato. Lo stimatore corretto e sempre valido della varianza \(\sigma^2\)
    è la Media dei Quadrati dell'Errore (\(MS_e\)):
    \[ \hat{\sigma}^2 = S_e^2 = MS_e = \frac{SS_e}{mn(l-1)} \quad \text{dove}
    \quad SS_e = \sum_{i,j,k} R_{ijk}^2 \]
\end{itemize}
L'analisi completa, che scompone la varianza totale, è riassunta nelle tabelle
seguenti.

\begin{table}[ht]
    \centering
    \caption{Tabella ANOVA a due fattori (caso con repliche) - Scomposizione
    della Varianza}
    \begin{tabular}{l l l}
        \toprule
        \textbf{Fonte di variabilità} & \textbf{Somma di quadrati (SS)} &
        \textbf{Gradi di libertà (df)} \\
        \midrule
        Riga (Fattore 1) & \(SS_r = nl \sum (\bar{Y}_{i**} - \bar{Y}_{***})^2\)
        & \(m-1\) \\
        Colonna (Fattore 2) & \(SS_c = ml \sum (\bar{Y}_{*j*} -
        \bar{Y}_{***})^2\) & \(n-1\) \\
        Interazione & \(SS_{int} = l \sum (\bar{Y}_{ij*} - \bar{Y}_{i**} -
        \bar{Y}_{*j*} + \bar{Y}_{***})^2\) & \((m-1)(n-1)\) \\
        Errore & \(SS_e = \sum (Y_{ijk} - \bar{Y}_{ij*})^2\) & \(mn(l-1)\) \\
        \bottomrule
    \end{tabular}
\end{table}

\begin{table}[ht]
    \centering
    \caption{Tabella ANOVA a due fattori (caso con repliche) - Test di Ipotesi}
    \resizebox{\textwidth}{!}{
    \begin{tabular}{l l l l}
        \toprule
        \textbf{Ipotesi Nulla} & \textbf{Statistica del Test} & \textbf{Un test
        con significatività \(\alpha\) deve...} & \textbf{p-dei-dati se
        \(F=v\)} \\
        \midrule
        \(H_0^{int}\): Le \(\gamma_{ij}\) sono tutte nulle & \(F_{int} =
        \frac{SS_{int}/((m-1)(n-1))}{SS_e/(mn(l-1))}\) & rifiutare \(H_0\) se
        \(F_{int} > F_{\alpha, (m-1)(n-1), mn(l-1)}\) & \(P(F_{(m-1)(n-1),
        mn(l-1)} \ge v)\) \\
        \(H_0^r\): Le \(\alpha_i\) sono tutte nulle & \(F_r =
        \frac{SS_r/(m-1)}{SS_e/(mn(l-1))}\) & rifiutare \(H_0\) se \(F_r >
        F_{\alpha, m-1, mn(l-1)}\) & \(P(F_{m-1, mn(l-1)} \ge v)\) \\
        \(H_0^c\): Le \(\beta_j\) sono tutte nulle & \(F_c =
        \frac{SS_c/(n-1)}{SS_e/(mn(l-1))}\) & rifiutare \(H_0\) se \(F_c >
        F_{\alpha, n-1, mn(l-1)}\) & \(P(F_{n-1, mn(l-1)} \ge v)\) \\
        \bottomrule
    \end{tabular}}
\end{table}

\begin{teorema}{Test F per il modello con
interazione}{anova-f-test-2way-int-thm}
La procedura di test è gerarchica. Il test per l'interazione (\(H_0:
\gamma_{ij} = 0\)) è il primo e più importante da valutare.
\[ F_{int} = \frac{MS_{int}}{S_e^2} \sim F((m-1)(n-1), mn(l-1)) \]
Se l'interazione non è significativa, si procede a testare gli effetti
principali.
\begin{itemize}
    \item \textbf{Righe}: \( F_r = \frac{MS_r}{S_e^2} \sim F(m-1, mn(l-1))\)
    \item \textbf{Colonne}: \( F_c = \frac{MS_c}{S_e^2} \sim F(n-1, mn(l-1))\)
\end{itemize}
\end{teorema}
\begin{nota}{Semplificazione del Modello}{model-simplification-note}
Se l'esito del test per uno degli effetti principali (es. righe) porta ad
accettare l'ipotesi nulla, si può considerare di eliminare quella variabile dal
modello, riducendo di fatto l'analisi a un'ANOVA a una via sul fattore
rimanente.
\end{nota}



\section{Il Test del Chi-Quadrato}

Il test del Chi-Quadrato (\(\chi^2\)) è una delle famiglie più importanti di
test statistici. Viene utilizzato principalmente come \textbf{test di
adattamento} (\textit{goodness-of-fit}) per verificare se una distribuzione di
dati osservata si adegua a una distribuzione teorica attesa.

Le principali applicazioni includono:
\begin{itemize}
    \item \textbf{Test elementare}: per distribuzioni discrete con un numero
    limitato di categorie (es. testare l'equità di un dado).
    \item \textbf{Generalizzazione}: per distribuzioni con molti valori o
    continue.
    \item \textbf{Test di adattamento a famiglie di distribuzioni}: per
    verificare se i dati seguono una certa famiglia di distribuzioni (es.
    Gaussiana) senza specificarne i parametri.
    \item \textbf{Tabelle di contingenza}: per verificare l'indipendenza tra due
    variabili categoriali.
\end{itemize}

\subsection{Test del Chi-Quadrato Elementare (Goodness-of-Fit)}
Questo test è il caso base e si applica quando si vuole confrontare la
distribuzione di un campione con una specifica distribuzione discreta.

\paragraph{Scopo e Ipotesi del Test}
Si parte da un campione \(X_1, \dots, X_n\) proveniente da una distribuzione
incognita \(\phi\). L'obiettivo è verificare se i dati si conformano a una
distribuzione "candidata" \(\phi_0\) completamente specificata.
\begin{itemize}
    \item \textbf{Ipotesi Nulla (\(H_0\))}: la distribuzione del campione è
    quella candidata.
    \[ H_0: \phi = \phi_0 \]
    \item \textbf{Ipotesi Alternativa (\(H_1\))}: la distribuzione del campione
    è diversa da quella candidata.
    \[ H_1: \phi \neq \phi_0 \]
\end{itemize}

\paragraph{Frequenze Osservate e Attese}
La procedura si basa sul confronto tra le frequenze osservate nel campione e
quelle che ci aspetteremmo se l'ipotesi nulla fosse vera.
\begin{itemize}
    \item \textbf{Frequenze Osservate (\(O_j\))}: si conta quante volte ciascuna
    delle \(k\) categorie possibili appare nel campione.
    \[ O_j = \text{numero di volte in cui è uscito il valore } j \]
    \item \textbf{Frequenze Attese (\(A_j\))}: sono le frequenze che ci
    aspetteremmo in media se i dati seguissero la distribuzione \(\phi_0\). Si
    calcolano come:
    \[ A_j = n \cdot p_{0,j} \]
    dove \(n\) è la numerosità del campione e \(p_{0,j}\) è la probabilità
    della categoria \(j\) secondo \(\phi_0\).
\end{itemize}

\paragraph{La Statistica del Test}
Per misurare la discrepanza tra le frequenze osservate e quelle attese, si usa
la statistica Chi-Quadrato di Pearson.

\begin{definizione}{Statistica Chi-Quadrato di Pearson}{pearson-chi2-def}
La statistica del test, indicata con \(W\) o \(\chi^2\), è calcolata come:
\[ W = \sum_{j=1}^{k} \frac{(O_j - A_j)^2}{A_j} \]
Questa statistica è "piccola" quando i dati osservati sono vicini a quelli
attesi (supportando \(H_0\)) e "grande" quando c'è una forte discrepanza
(supportando \(H_1\)).
\end{definizione}

\begin{teorema}{Distribuzione della Statistica del Test}{chi2-dist-thm}
Sotto l'ipotesi nulla \(H_0\), la statistica del test \(W\) segue
asintoticamente (per \(n\) grande) una \textbf{distribuzione Chi-Quadrato con
\(k-1\) gradi di libertà}:
\[ W \stackrel{H_0}{\sim} \chi^2_{k-1} \]
Poiché valori grandi della statistica forniscono evidenza contro \(H_0\), il
test è \textbf{unilaterale destro}.
\end{teorema}

\begin{nota}{Condizioni di Validità e Potenza}{chi2-validity-note}
\begin{itemize}
    \item \textbf{Regola Pratica}: l'approssimazione alla distribuzione
    \(\chi^2\) è considerata valida se tutte le frequenze attese sono \(A_j \ge
    1\) e almeno l'80\% di esse sono \(A_j \ge 5\).
    \item \textbf{Potenza}: il test del Chi-Quadrato è noto per essere poco
    potente. Richiede scostamenti notevoli o campioni molto grandi per rigettare
    \(H_0\) quando le differenze tra \(\phi\) e \(\phi_0\) sono modeste.
    \item \textbf{Statistica G}: negli ultimi anni si tende a preferire la
    statistica G (o del rapporto di verosimiglianza), che segue la stessa
    distribuzione \(\chi^2_{k-1}\) ma con un'approssimazione spesso migliore:
    \[ G = 2 \sum_{j=1}^{k} O_j \ln\left(\frac{O_j}{A_j}\right) \]
\end{itemize}
\end{nota}

\subsection{Estensioni del Test del Chi-Quadrato}
Il test elementare può essere generalizzato per affrontare due scenari più
complessi: l'adattamento a distribuzioni continue e l'adattamento a famiglie di
distribuzioni con parametri non specificati.

\paragraph{Test di Adattamento per Leggi Continue}
Quando la distribuzione candidata \(\phi_0\) è continua (es. Esponenziale,
Normale, etc.), non è possibile contare le occorrenze di singoli valori. La
procedura viene quindi adattata "discretizzando" il supporto della
distribuzione.

\begin{enumerate}
    \item \textbf{Binning}: Si suddivide il dominio della variabile in \(k\)
    intervalli (o \textit{bin}) disgiunti, \(B_1, B_2, \dots, B_k\).
    \item \textbf{Frequenze Osservate}: Si contano quante osservazioni del
    campione cadono in ciascun bin per ottenere le frequenze osservate \(O_j\).
    \item \textbf{Frequenze Attese}: Si calcola la probabilità \(p_{0,j}\) che
    un'osservazione cada nel bin \(j\) secondo la distribuzione nulla
    \(\phi_0\). Per una distribuzione continua con densità \(f_0(x)\), questo
    si ottiene integrando:
    \[ p_{0,j} = P(X \in B_j) = \int_{B_j} f_0(x) dx \]
    Le frequenze attese sono quindi \(A_j = n \cdot p_{0,j}\).
\end{enumerate}
Una volta ottenute le frequenze \(O_j\) e \(A_j\), si procede con la statistica
di Pearson \(W\) e il confronto con la distribuzione \(\chi^2_{k-1}\)
esattamente come nel caso elementare.

\begin{nota}{Strategia di Binning}{binning-strategy-note}
A parità di \(k\), il test è più potente e l'approssimazione alla \(\chi^2\)
è migliore se le frequenze attese \(A_j\) sono più grandi possibile. Per
questo motivo, la strategia ottimale non è creare bin di larghezza uguale, ma
\textbf{bin equiprobabili}, ovvero intervalli scelti in modo tale che la
probabilità \(p_{0,j}\) (e quindi la frequenza attesa \(A_j\)) sia la stessa
per ogni bin. La potenza del test è comunque legata al numero e alla posizione
dei bin scelti.
\end{nota}

\paragraph{Test di Adattamento a Famiglie di Distribuzioni}
Spesso non si vuole testare l'adattamento a una distribuzione
\textit{completamente} specificata, ma a una \textit{famiglia} di distribuzioni,
lasciando i parametri liberi. Un esempio classico è il test di Gaussianità.
\begin{itemize}
    \item \textbf{Ipotesi}: \(H_0: X \sim N(\mu, \sigma^2)\) con \(\mu\) e
    \(\sigma^2\) non specificati.
\end{itemize}
La procedura è la seguente:
\begin{enumerate}
    \item \textbf{Stima dei Parametri}: Si stimano i parametri incogniti della
    distribuzione a partire dal campione (es. \(\hat{\mu} = \bar{x}\) e
    \(\hat{\sigma}^2 = s^2\)).
    \item \textbf{Test Standard}: si esegue il test di adattamento del
    Chi-Quadrato (con la procedura di binning vista sopra) usando come
    distribuzione nulla \(\phi_0\) quella della famiglia specificata, ma con i
    parametri appena stimati (es. \(N(\bar{x}, s^2)\)).
    \item \textbf{Correzione dei Gradi di Libertà}: la distribuzione della
    statistica del test \(W\) è ancora una Chi-Quadrato, ma i suoi gradi di
    libertà vengono ridotti.
\end{enumerate}

\begin{teorema}{Gradi di Libertà con Parametri
Stimati}{df-estimated-params-thm}
Se si stimano \(s\) parametri dal campione per definire la distribuzione nulla
\(\phi_0\), la statistica del test \(W\) si distribuisce come una Chi-Quadrato
con \(k-1-s\) gradi di libertà.
\[ W \stackrel{H_0}{\sim} \chi^2_{k-1-s} \]
Nel caso del test di Gaussianità, si stimano \(s=2\) parametri (\(\mu\) e
\(\sigma^2\)), quindi i gradi di libertà sono \(k-3\).
\end{teorema}

\begin{nota}{Uso Pratico dei Test di
Adattamento}{goodness-of-fit-practical-note}
L'uso di test formali per verificare a priori o a posteriori le assunzioni di un
modello (es. la Gaussianità dei residui) non è sempre una buona idea. Con
campioni molto grandi, questi test diventano "troppo potenti" e possono
rigettare l'ipotesi nulla per deviazioni minime e praticamente irrilevanti.
L'ipotesi che spesso ci interessa non è se i dati siano \textit{esattamente}
Normali, ma se siano \textit{sufficientemente vicini} alla Normalità perché le
nostre procedure funzionino. Per questo, è spesso preferibile affiancare o
sostituire il test formale con un \textbf{controllo qualitativo e grafico} (es.
un Q-Q plot).
\end{nota}



\subsection{Test del Chi-Quadrato per Tabelle di Contingenza}
Questa è una delle applicazioni più comuni del test del Chi-Quadrato e serve a
verificare se esista un'associazione tra due variabili categoriali. A differenza
della regressione o dell'ANOVA, questa analisi è simmetrica: non si definisce
una variabile di ingresso e una di uscita, ma si studia la relazione reciproca
\(X \leftrightarrow Y\).

\paragraph{Scopo e Ipotesi}
L'obiettivo del test è determinare se due variabili categoriali sono
statisticamente indipendenti o se sono associate.
\begin{itemize}
    \item \textbf{Ipotesi Nulla (\(H_0\))}: Le due variabili sono indipendenti.
    In termini di probabilità, la probabilità congiunta è il prodotto delle
    probabilità marginali: \( P(X=i, Y=j) = P(X=i) \cdot P(Y=j) \).
    \item \textbf{Ipotesi Alternativa (\(H_1\))}: Le due variabili non sono
    indipendenti (esiste un'associazione).
\end{itemize}
Si tratta di un test \textbf{non parametrico}, in quanto non fa alcuna ipotesi
sulla famiglia di distribuzione sottostante, il che lo rende meno potente di
test come l'ANOVA se le assunzioni di quest'ultima sono soddisfatte.

\paragraph{Frequenze Osservate e Attese}
Il test si basa sul confronto tra le frequenze congiunte osservate nel campione
e quelle che ci aspetteremmo in un'ipotetica situazione di indipendenza.
\begin{itemize}
    \item \textbf{Frequenze Osservate (\(O_{ij}\))}: sono i conteggi effettivi
    per ogni cella della tabella di contingenza, ovvero il numero di
    osservazioni che presentano simultaneamente la categoria \(i\) della prima
    variabile e la categoria \(j\) della seconda.
    \item \textbf{Frequenze Attese (\(A_{ij}\))}: sono i conteggi che ci
    aspetteremmo in ogni cella se \(H_0\) (indipendenza) fosse vera. Poiché le
    probabilità marginali sono incognite, vengono stimate dai dati. La formula
    per le frequenze attese è:
    \[ A_{ij} = \frac{(\text{Totale della riga } i) \cdot (\text{Totale della
    colonna } j)}{\text{Totale generale}} \]
\end{itemize}

\begin{definizione}{Statistica del Test per l'Indipendenza}{indep-test-stat-def}
La statistica del test è la consueta statistica Chi-Quadrato di Pearson,
calcolata su tutte le \(m \times n\) celle della tabella:
\[ W = \sum_{i=1}^{m} \sum_{j=1}^{n} \frac{(O_{ij} - A_{ij})^2}{A_{ij}} \]
\end{definizione}

\begin{teorema}{Distribuzione e Gradi di Libertà}{chi2-contingency-thm}
Sotto l'ipotesi nulla di indipendenza, la statistica del test \(W\) segue
asintoticamente una distribuzione Chi-Quadrato con \((m-1)(n-1)\) gradi di
libertà, dove \(m\) è il numero di righe e \(n\) è il numero di colonne.
\[ W \stackrel{H_0}{\sim} \chi^2_{(m-1)(n-1)} \]
\end{teorema}

\begin{dimostrazione}{Calcolo dei Gradi di Libertà}{dof-contingency-dem}
La regola generale per i gradi di libertà è:
\[
gdl = (\text{numero categorie}) - 1 - (\text{numero parametri stimati})
\]
\begin{itemize}
    \item Numero totale di categorie: \(m \cdot n\).
    \item Per stimare le frequenze attese, abbiamo dovuto stimare le
    probabilità marginali. Per le \(m\) righe, i parametri liberi sono \(m-1\)
    (poiché la somma deve fare 1). Per le \(n\) colonne, i parametri liberi
    sono \(n-1\).
    \item Numero totale di parametri stimati: \((m-1) + (n-1)\).
\end{itemize}
Quindi: \( gdl = (m \cdot n) - 1 - [(m-1) + (n-1)] = mn - 1 - m + 1 - n + 1 = mn
- m - n + 1 = (m-1)(n-1) \).
\end{dimostrazione}

\begin{nota}{Note Pratiche}{contingency-practical-notes}
\begin{itemize}
    \item \textbf{Variabili Continue}: Se una o entrambe le variabili non sono
    discrete con poche categorie, devono essere prima "discretizzate" tramite
    binning.
    \item \textbf{Condizioni di Validità}: Anche in questo caso,
    l'approssimazione alla \(\chi^2\) è valida se le frequenze attese non sono
    troppo piccole (la regola pratica è che l'80\% delle celle abbia \(A_{ij}
    \ge 5\) e tutte abbiano \(A_{ij} \ge 1\)).
\end{itemize}
\end{nota}


\section{Versioni Esatte dei Test del Chi-Quadro}
L'approssimazione della statistica \(W\) con una distribuzione \(\chi^2\) è un
risultato asintotico, valido per grandi campioni. Quando le frequenze attese
sono basse e la "rule of thumb" non è rispettata, questa approssimazione non è
più affidabile. In questi casi, si ricorre a versioni esatte del test, che
calcolano la distribuzione della statistica del test sotto \(H_0\) senza fare
affidamento sull'approssimazione.

Questo si può ottenere in due modi:
\begin{itemize}
    \item \textbf{Metodo esatto (esaustione)}: tramite calcolo combinatorio, si
    enumerano tutti i possibili risultati e si calcola la probabilità esatta di
    ciascuno. È fattibile solo per problemi di piccole dimensioni.
    \item \textbf{Simulazione Monte Carlo (MCS)}: si genera un gran numero di
    campioni casuali dalla distribuzione nulla \(\phi_0\), si calcola la
    statistica del test per ciascuno e si costruisce una distribuzione empirica,
    che approssima quella vera.
\end{itemize}

\subsection{Test Esatto di Goodness-of-Fit}
Supponiamo di voler eseguire un test di adattamento, ma le frequenze attese sono
troppo piccole.

\begin{esempio}{Test esatto per "titolo di studio"}{exact-gof-example}
Consideriamo un campione di \(n=20\) studenti e vogliamo testare se la
distribuzione del loro titolo di studio si conforma a una \(\phi_0\) data. Se le
frequenze attese calcolate \(A_j = n \cdot p_{0,j}\) risultano essere, ad
esempio, \([2, 10, 6, 2]\), la regola pratica per l'uso della \(\chi^2\) non è
rispettata. Non possiamo quindi fidarci del p-value ottenuto da una
distribuzione \(\chi^2_3\). Per calcolare il p-value corretto, dobbiamo trovare
la vera distribuzione della statistica \(W\) sotto \(H_0\). Con un metodo di
simulazione Monte Carlo, ad esempio, possiamo generare 100.000 campioni da
\(\phi_0\), calcolare \(W\) per ciascuno, e usare la distribuzione empirica di
questi valori per calcolare il p-value.
\end{esempio}

\subsection{Il Test Esatto di Fisher per Tabelle 2x2}
Il test esatto di Fisher è la soluzione combinatoria esatta per il test di
indipendenza in una tabella di contingenza 2x2. È particolarmente utile quando
le numerosità campionarie sono piccole.

\paragraph{L'Idea Fondamentale}
Invece di usare la statistica \(W\), il test si concentra su una delle quattro
celle della tabella, tipicamente quella in alto a sinistra (\(O_{11}\)). L'idea
chiave è che, una volta \textbf{fissati i totali marginali} della tabella, il
valore di una singola cella determina i valori di tutte le altre.

\begin{table}[ht]
    \centering
    \caption{Tabella di contingenza 2x2 con notazione dei marginali.}
    \begin{tabular}{c|cc|c}
        & Colonna 1 & Colonna 2 & Totale Riga \\
        \hline
        Riga 1 & \(O_{11}\) & \(O_{12}\) & \(c\) \\
        Riga 2 & \(O_{21}\) & \(O_{22}\) & \(d\) \\
        \hline
        Totale Colonna & \(a\) & \(b\) & \(n\) \\
    \end{tabular}
\end{table}

Il problema è analogo a un'estrazione da un'urna. Immaginiamo un'urna con \(n\)
palline, di cui \(a\) nere (colonna 1) e \(b\) bianche (colonna 2). Se estraiamo
\(c\) palline (la prima riga), il numero di palline nere estratte (\(k\))
corrisponde al valore di \(O_{11}\). Questa logica porta direttamente alla
distribuzione di probabilità esatta.

\begin{teorema}{Distribuzione per il Test di Fisher}{fisher-test-dist-thm}
Assumendo l'ipotesi nulla \(H_0\) di indipendenza tra le due variabili e
condizionando ai totali marginali, la probabilità di osservare un valore \(k\)
nella cella \((1,1)\) di una tabella 2x2 segue la \textbf{distribuzione
Ipergeometrica}:
\[ P(O_{11}=k | H_0, \text{marginali}) = \frac{\binom{a}{k}
\binom{b}{c-k}}{\binom{n}{c}} \]
\end{teorema}

\begin{esempio}{Calcolo della distribuzione esatta}{fisher-example}
    Consideriamo una tabella 2x2 con totali marginali: \(c=8\) (riga 1),
    \(a=21\) (colonna 1), \(b=21\) (colonna 2), e \(n=42\). Sotto \(H_0\), la
    probabilità per ogni possibile valore di \(O_{11}\) (da 0 a 8) è data
    dalla distribuzione Ipergeometrica. La tabella seguente e il grafico a barre
    mostrano questa distribuzione.

    \centering
    \begin{tabular}{c c}
        \toprule
        \textbf{k} & \textbf{P(\(O_{11}=k\))} \\
        \midrule
        0 & 0.00172 \\
        1 & 0.02069 \\
        2 & 0.09655 \\
        3 & 0.22930 \\
        4 & 0.30348 \\
        5 & 0.22930 \\
        6 & 0.09655 \\
        7 & 0.02069 \\
        8 & 0.00172 \\
        \bottomrule
    \end{tabular}
\end{esempio}

\paragraph{Calcolo del p-value}
Il test è tipicamente bilaterale. Il p-value si calcola sommando le
probabilità di tutti i risultati altrettanto o più "estremi" (rari) di quello
osservato. Se la distribuzione è simmetrica, come nell'esempio sopra, e
osserviamo \(O_{11}=1\), il p-value sarà la somma delle probabilità nelle due
code: \( P(O_{11} \le 1) + P(O_{11} \ge 7) \approx 0.02069 + 0.00172 + 0.02069 +
0.00172 = 0.0448 \).
